\section{شرح مسأله و چشم انداز پروژه}
با توجه به پیشرفت تکنولوژی و در نتیجه افزایش ارتباطات  و همچنین گردش  اطلاعات تسلط به زبان انگلیسی به یک نیاز تبدیل شده است.  می‌توان گفت امروزه دانش آموزان، دانشجویان و تمام کسانی که می‌خواهند به بازار کار بپیوندند نیاز دارند تا به زبان انگلیسی مسلط باشند. در فرآیند یادگیری زبان، مهم است که ارتباطی پیوسته بین زبان آموز و زبان مورد نظر وجود داشته باشد. یکی از راه‌حل هایی که توسط معلمان زبان نیز به کار گرفته می‌شود استفاده از فیلم‌های سینمایی به زبان مورد نظر  است. در این رویکرد زبان آموز یک فیلم (به زبان مورد نظر مثلا انگلیسی) را به همراه زیرنویس آن (به زبان اصلی فیلم) تماشا می‌کند. در حین تماشای فیلم زبان آموز کلماتی که با آن‌ها آشنا نیست را یادداشت می‌کند و پس از اتمام فیلم سعی می‌کند که کلمات مورد نظر را فرا بگیرد. با استفاده از این رویکرد زبان آموز علاوه بر آشنایی با لغات با نحوه به کارگیری آن‌ها در جملات و تلفظ آن‌ها نیز آشنا می‌شود. 
 با توجه به کم هزینه بودن تماشای فیلم در کشورمان یادگیری زبان به کمک فیلم‌های سینمایی یکی از به صرفه‌ترین، جذاب‌ترین و محبوب‌ترین راه‌های یادگیری زبان است.
 
 متاسفانه این روش نقاط ضعفی نیز دارد. در این رویکرد  زبان آموز مجبور است تا مدام فیلم را متوقف کند و کلمات را یادداشت کند. همچنین ممکن است دانستن معنی برخی از آن کلمات برای درک فیلم ضروری باشد و زبان آموز مجبور شود تا در همان موقع به دنبال معنی کلمه مورد نظر نیز بگردد.  این موضوع باعث می‌شود این رویکرد وقت‌‌گیر شود و از جذابیت آن نیز کاسته شود و در نتیجه از استفاده آن کاسته شود. نکته دیگر این است که یادداشت کلمات بدون مثال استفاده از آن‌ها معمولا تاثیر کمتری  در یادگیری خواهد داشت. در حقیقت برای یادگیری بهتر می‌بایست علاوه بر یادداشت کلمه، یک جمله که آن کمله در آن به کار رفته است (مثلا جمله فیلم) نیز یادداشت شود که اگر کسی بخواهد این کار را نیز انجام دهد مجددا وقت بیشتری از او گرفته خواهد شد.
 
 محصولی که ارائه می‌دهیم تمام این مشکلات را حل می‌کند. پروژه مورد نظر ما یک سایت برای یادگیری زبان انگلیسی به کمک فیلم است. برای استفاده از سایت زبان آموز ابتدا یک حساب در سایت تشکیل می‌دهد و  در یک تعیین سطح سریع از واژگان\efn{Vocabulary}
 زبان انگلیسی شرکت می‌کند.
 در ادامه برای هر فیلمی که تماشا می‌کند زیرنویس انگلیسی آن را در سایت آپلود می‌کند. با توجه به برآوردی که سایت از سطح زبان او دارد، تعدادی از واژگان برای یادگیری او مناسب تشخیص داده می‌شوند. این واژگان مشخص شده ترجمه می‌شوند و ترجمه آن‌ها در زیرنویس انگلیسی قرار می‌گیرد. با این کار زبان آموز می‌تواند در حین تماشای فیلم با معنی واژگان مورد نظر آشنا شود. همچنین سایت یک مجموعه فلش کارت\efn{Flashcard}
 از این واژگان می‌سازد. بر روی هر کدام از این فلش کارت‌ها معنی کلمه مورد نظر (به عنوان جواب) و جمله‌ای از فیلم که این کلمه در آن به کار رفته است (به عنوان راهنمایی) قرار می‌گیرد. پس از تماشای فیلم زبان آموز می‌تواند به یادگیری کلمات و دادن آزمون از آن‌ها بپردازد (مانند سایر سرویس‌های فلش کارت). همچنین سایت برآورد خود از زبان آموز را بهبود می‌بخشد و هیچگاه کلمات تکراری (که یاد گرفته است) را برای او انتخاب نمی‌کند. با این پروژه در حین تماشای فیلم زمان زبان آموز برای یادداشت و معنی کردن کلمات گرفته نمی‌شود. همچنین زبان آموز در انتهای فیلم به کلمات مهم  به همراه مثال (جمله‌ای که در فیلم به کار رفته) دسترسی دارد که فرآیند یادگیری وی را بهبود می‌بخشد. پروژه ما  علاوه بر این که باعث یادگیری لغات فیلم توسط زبان آموز می‌شود لذت فیلم دیدن را کاهش نمی‌دهد و انگیزه و اشتیاق زبان آموز را حفظ می‌کند و به همین دلیل می‌تواند به عنوان راه‌حلی کم هزینه و جذاب  برای یادگیری زبان استفاده شود.
 
 \section{نیازمندی‌های پروژه}
 در این قسمت به بررسی نیازمندی‌های\efn{Requirement}
 پروژه هم از نظر 
 کارکردی\efn{Functional}
  و هم از نظر غیر کارکردی\efn{Non-functional}
   می‌پردازیم.
  \subsection{نیازمندی‌های کارکردی}
  \begin{itemize}
  	\item نیازمندی‌های حساب کاربری
  	\begin{enumerate}
  	\item امکان ساخت حساب کاربری در سایت
  	\item امکان دادن آزمون تعیین سطح
  	\item امکان شارژ کردن حساب کاربری
  	\end{enumerate}
  	\item نیازمندی‌های مربوط به زیرنویس
  	\begin{enumerate}
  	\item امکان تشخیص کلمات مهم زیرنویس با توجه به سطح زبان آموز
  	\item امکان دریافت (آپلود شدن) زیرنویس، معنی کردن کلمات انتخاب شده در زیرنویس و دانلود زیر نویس اصلاح‌شده
  	\item امکان افزودن کلمات انتخاب شده به همراه معنی و جمله فیلم به فلش‌ کارت‌ها
  	\end{enumerate}
  	\item نیازمندی‌های مربوط به فلش کارت‌ها
  	\begin{enumerate}
  	\item امکان انجام فعالیت یادگیری فلش کارت‌ها با توجه به زمان انتخاب شده توسط کاربر
  	\item امکان برگزاری آزمون از فلش کارت‌ها
  	\item  امکان تعیین میزان یادگیری یک کلمه توسط زبان آموز (تا بتوان فرآیند یادگیری را متناسب با آن انجام داد)
  	\item امکان بروزرسانی سطح کاربر با توجه به فعالیت‌های یادگیری و آزموننیازمندی 
  	\end{enumerate}
  \end{itemize} 
  \subsection{نیازمندی‌های غیر کارکردی}
  در اینجا مهم‌ترین نیازمندی‌های غیر کارکردی سیستم را بیان می‌کنیم.
  \begin{itemize}
  	\item مقیاس پذیری\efn{Scalability} (برای پشتیبانی از تعداد کاربران بالا)
  	\item قابلیت حمل\efn{Portability} (برای سازگار بودن با مرورگر‌ها مختلف بر روی کامپیوتر و موبایل)
  	\item قابلیت استفاده\efn{Usability} (داشتن قابلیت‌هایی مانند یادگیری سریع\efn{Learnability})
  	\item امنیت\efn{Security} (به خصوص در حوزه شارژ کردن حساب‌ها) 
  	\item دسترس پذیری\efn{Availability} (البته این نیاز‌مندی در کنار  قابلیت اطمینان\efn{Reliability} و نگه‌داشت پذیری\efn{Maintainability} معنی می‌دهد)
  \end{itemize}

\section{محدودیت‌های پروژه}
در خصوص خود پروژه به نظر نمی‌رسد تا با محدودیت خاصی سرکار داشته باشیم. توجه کنیم که پردازشی که بر روی سامانه‌مان انجام می‌گیرد و داده‌ای که ذخیره می‌کنیم چندان سنگین نیست (البته کمی بستگی به سیستم‌هایی که برای ارزیابی میزان سختی کلمات استفاده  می‌کنیم دارد). همچنین با توجه به این که تنها از زیرنویسی که کاربر  آپلود می‌کند استفاده می‌کنیم به نظر نمی‌رسد مشکلات قانونی (به خصوص با توجه به قوانین  ایران در زمینه حق نشر\efn{Copyright})‌ داشته باشیم. برای همین تنها چند محدودیت کلی وجود دارد که به آن‌ها اشاره می‌کنیم.
\begin{itemize}
	\item محدودیت هزینه و منابع: با توجه به این محدودیت بررسی مواردی مانند مقیاس پذیری سامانه سخت‌تر می‌شود.
	\item زمان پروژه: این مورد نیز می‌تواند بر روی کیفیت پروژه تأثیر داشته باشد. برای مثال با زمان بیشتر می‌توان از روش‌های پیشرفته‌تر یادگیری ماشین برای سطح‌بندی کاربر و کلمات استفاده کرد.
	\item مورد دیگری که باید در پروژه به آن توجه داشت لایسنس‌ منابعی است که برای تعیین سطح کلمات و کاربر انتخاب می‌کنیم (برخی موارد تنها در صورت استفاده تحقیقاتی رایگان هستند). همچنین این موارد باید سرعت مناسبی در تعیین سطح کلمه داشته باشند تا پردازش زیرنویس سریع انجام شود.
\end{itemize} 